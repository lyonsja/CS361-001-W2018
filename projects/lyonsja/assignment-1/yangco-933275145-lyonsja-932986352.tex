\documentclass[12pt]{article}
\usepackage{url}




\title{Textbook Buyback App--Fairbook}

\author{Cong Yang, Jacob Lyons}



\begin{document}
	\maketitle

	\tableofcontents
                \section{Info}
		Info \par
		Team Name: Fairbook \par
		Cong Yang ONID: yangco \par
		Jacob Lyons ONID: lyonsja \par


	\section{Problem}
	The price of textbooks continues to rise unreasonably fast--82\% between 2002 and 2013.
	\cite{cnbc}
	Much of this issue can be clearly attributed to underhanded business practices undertaken
	by publishers. Mainly, releasing unecessary new editions of existing textbooks in order to maintain
	profits. 
	However, publishers aren't the only ones taking advantage of students. 
	Existing textbook buyback options for students are sorely lacking. With the demise of Half.com, 
	no good replacement has appeared for students looking to sell their textbooks quickly--and without 
	getting ripped off.	
        Most buyback programs use predatory pricing tactics that blatantly exploit students. Although independent options 
	exist (ebay, craigslist, Amazon, etc.), they are
	generalized marketplaces with no special consideration for students just trying to sell books. For example,
	ebay places limits on how many items new sellers can list. 
	Amazon has a more involved process before allowing accounts to start selling items.
	Bookscouter may provide some similar function to this app, but it simply aggregates existing buyback programs and is not itself
	a buyback service.
 	For students just trying to sell a few books,
	there are almost no good options.\par
	
	When the time comes to sell back textbooks, students are mostly presented with buyback programs heavily weighted
	against them. My friend has been unable to sell her textbook for anything close to a fair price--even though it was only published 8 months prior.
	All of the services she has looked into are offering only a tenth of the book's actual value. The bookstore pays only
	a fractional value of the textbook as do the buyback tents that appear seasonally. \par
		

	\section{Solution}

	The solution is to create a new web (and/or mobile) app that addresses the issues with existing platforms. Mainly the
	exploitave pricing and lack of a dedicated, local textbook buying/selling service.
	It will ideally target the college student demographic as a robust, accessible app. Utilizing existing online
        information on textbook prices (Amazon, ebay, etc), it must have \underline{fair} pricing for students--closer to a 10\% cut or a scaling cut. 
	This pricing element would be one of the key difference between this app and existing services--most of which offer veritable pennies for
	textbooks in great condition.
	\subsection{Features}
		\subsubsection{Basic Features}
		Fairbook should have functionality for students to add pictures, title of, and condition of the book they are trying to sell.
		This app should have an interface for both buyers and sellers. Interface should be intuitive and minimalistic.
		\subsubsection{Search}
        	Search feature is essential to this app. As is the ability to sort and filter books by distance, condition, price, and ISBN. This would be the minimum
		search functionality required.
		\subsubsection{Local Meetup Feature}
		The app should have functionality for students to connect locally and meet up in person to sell textbooks.
		This could be in the form of DM from within the app itself, or simply a way to exchange contact information between local buyers and sellers	
		\subsubsection{Shipping Functionality}
		This app would place burden of shipping mostly on the individual. A way to export buyer's address for label creation
		purposes will be necessary. A shipping cost calculator could also be implemented. This would take into account buyer and seller location
		relative to one another and adjust accordingly. 

	\section{Risks}
	
		\subsection{Initial User Base}
		The major risk of this application would be that of gaining a large user base (via marketing,etc.).
        	Even if the app itself is well-developed, it will require a user base in order to be effective. The local
		buying/selling portion of the service may help it get off the ground.	
		Perhaps some sort of initial incentive could be afforded for early users. This could take the form of "free"
		listings granted to new users. 
		\subsection{Competitiveness}
		Another risk would be that of simply being uncompetitive in relation to the existing (flawed)
		services. However, if (as intended) pricing for the books is more reasonable than other platforms, this risk
		could be mitigated.	


\bibliographystyle{plain}
\bibliography{yangco-933275145-lyonsja-932986352}
\end {document}


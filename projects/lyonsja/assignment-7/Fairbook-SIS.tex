\documentclass[12pt]{article}
\usepackage{url}

\title{Fairbook: FIS}

\author{A. Snow, C. McKenzie, C. Yang, J. Lyons, M. Zhang}

\begin{document}
	\maketitle


	\tableofcontents
	\section{ONIDs}
		snowan, mckencod, yangco, lyonsja, zhangm4



	\section{Product Release}
	We have our web app code in a github repository with the URL: \url{https://github.com/lyonsja/CS361-Group14/tree/master}.
	All that needs to be done is downloading the component files and opening index.html in your web browser (we only tested using Google Chrome). 
	From there, it is possible to access the about page and detailed book view page from their respective links. In addition, the user can also follow the seller link on the
	Navigation Bar and fill out the form in order to post a new book to the main page.
	The user can also return to the home page. 
	There shouldn’t be any need to install additional software or libraries. However, some of the methods used in this project
	are only compatible with HTML 5, so that is one consideration to take into account.

	\section{User Story}
		\subsection{Creating the shopping cart}
		\begin{iterate}

		\item The teammates working here are Andrew(HTML), Cong(CSS) and Jacob(JS) with Mingyu and Cody as backup for reviewing and error checking.

		\item The way we are working on this is through a Github repository that we push the code to (and pull from in order to view the changes and or make necessary changes).  We have been communicating through Discord to let each other know what all has been done (we found discord to be the best way to communicate with one another in a timely manner).  The reason we have Andre working on HTML, Cong on CSS and Jacob with JS is so we won’t have multiple people changing the code others are working on and screwing up the repository by accident.  They individually work on their code and get it ready and then let Mingyu and I know when they are done so we can look over the code and make sure all is working properly.  The only problems we have encountered (besides the fact this big project is expected to be done in a very short time frame that is unrealistic) is scheduling with one another and our work around is using Discord.

		\item Each task took about 1-3 hours to accomplish.

		\item Finished.

		\item Diagrams have not been entirely useful because this project is pretty straightforward, we all know what we need to accomplish and how we will.
		\end{iterate}

		\subsection{Adding books to the cart}
		\begin{iterate}

		\item The teammates working here are Andrew(HTML), Cong(CSS) and Jacob(JS) with Mingyu and Cody as backup for reviewing and error checking.

		\item The way we are working on this is through a Github repository that we push the code to (and pull from in order to view the changes and or make necessary changes).  We have been communicating through Discord to let each other know what all has been done (we found discord to be the best way to communicate with one another in a timely manner).  The reason we have Andre working on HTML, Cong on CSS and Jacob with JS is sso we won’t have multiple people changing the code others are working on and screwing up the repository by accident.  They individually work on their code and get it ready and then let Mingyu and I know when they are done so we can look over the code and make sure all is working properly.  The only problems we have encountered (besides the fact this big project is expected to be done in a very short time frame that is unrealistic) is scheduling with one another and our work around is using Discord.

		\item Each task took about 1-3 hours to accomplish.

		\item semi-Finished.

		\item Diagrams have not been entirely useful because this project is pretty straightforward, we all know what we need to accomplish and how we will.

		\end{iterate}

		\subsection{Removing books from the cart}
		\begin{iterate}

		\item The teammates working here are Andrew(HTML), Cong(CSS) and Jacob(JS) with Mingyu and Cody as backup for reviewing and error checking.

		\item The way we are working on this is through a Github repository that we push the code to (and pull from in order to view the changes and or make necessary changes).  We have been communicating through Discord to let each other know what all has been done (we found discord to be the best way to communicate with one another in a timely manner).  The reason we have Andre working on HTML, Cong on CSS and Jacob with JS is sso we won’t have multiple people changing the code others are working on and screwing up the repository by accident.  They individually work on their code and get it ready and then let Mingyu and I know when they are done so we can look over the code and make sure all is working properly.  The only problems we have encountered (besides the fact this big project is expected to be done in a very short time frame that is unrealistic) is scheduling with one another and our work around is using Discord.

		\item Each task took about 1-3 hours to accomplish.

		\item Finished

		\item Diagrams have not been entirely useful because this project is pretty straightforward, we all know what we need to accomplish and how we will.

		\end{iterate}

		\subsection{Account information}
		\begin{iterate}

		\item The teammates working here are Andrew(HTML), Cong(CSS) and Jacob(JS) with Mingyu and Cody as backup for reviewing and error checking.

		\item The way we are working on this is through a Github repository that we push the code to (and pull from in order to view the changes and or make necessary changes).  We have been communicating through Discord to let each other know what all has been done (we found discord to be the best way to communicate with one another in a timely manner).  The reason we have Andre working on HTML, Cong on CSS and Jacob with JS is sso we won’t have multiple people changing the code others are working on and screwing up the repository by accident.  They individually work on their code and get it ready and then let Mingyu and I know when they are done so we can look over the code and make sure all is working properly.  The only problems we have encountered (besides the fact this big project is expected to be done in a very short time frame that is unrealistic) is scheduling with one another and our work around is using Discord.

		\item Each task took about 1-3 hours to accomplish.

		\item Finished.

		\item Diagrams have not been entirely useful because this project is pretty straightforward, we all know what we need to accomplish and how we will.

		\end{iterate}

		\subsection{Adding books to the site}
		\begin{iterate}

		\item The teammates working here are Andrew(HTML), Cong(CSS) and Jacob(JS) with Mingyu and Cody as backup for reviewing and error checking.

		\item The way we are working on this is through a Github repository that we push the code to (and pull from in order to view the changes and or make necessary changes).  We have been communicating through Discord to let each other know what all has been done (we found discord to be the best way to communicate with one another in a timely manner).  The reason we have Andre working on HTML, Cong on CSS and Jacob with JS is sso we won’t have multiple people changing the code others are working on and screwing up the repository by accident.  They individually work on their code and get it ready and then let Mingyu and I know when they are done so we can look over the code and make sure all is working properly.  The only problems we have encountered (besides the fact this big project is expected to be done in a very short time frame that is unrealistic) is scheduling with one another and our work around is using Discord.

		\item Each task took about 1-3 hours to accomplish.

		\item Finished.

		\item Diagrams have not been entirely useful because this project is pretty straightforward, we all know what we need to accomplish and how we will.

		\end{iterate}



	\section{Design changes and rationale}
	




	\section{Refactoring}
		So far, we have work out a prototype for our project, and we come up several ideas to make it easier to maintaining.

	Here is what we will change in our program. First, we will break the HTML part into several templates and display our interface dynamically through Javascript. Second, we initially stored our information in local . json file because it is easy to track and implement, now we decide to use Mongodb as our database to store all information.


	\subsection{Dynamic Content}
	In our first version, we write the interface by HTML and CSS, it is no different compare with the change. But each item or element in the interface is including in the HTML file and the key words is surrounded by some HTML components, which makes the file quite large and hard to maintain because of its size.

	Also, the large size of the file makes the program looks more complex and easy to make error. So, we decide to show the interface dynamically.

	What we will do is firstly complete and test our current one to see if it works. Then break each identity element into one template for later usage. For example, we have a block of code to show a book’ information, we isolate this part and store it in a new file, so we could reuse it to show different boos’ information instead of writing same code for all books.

	After we break our HTML file into different templates, we change our Javascript file that shows the required content by template instead of a signal HTML file.

	By making the change, the function keeps same, but the structure of the program is more clear and makes it easy to maintain and modify.


	\subsection{Mongodb}
	 We firstly choose to store our information such as books, users in json file, which is easy to implement and track. But when the program has huge amount of data, it hard to maintain and manage those files. Also, those files are stored locally, which makes them easy to lose or mistakenly changed.

	So, we change the way storing information to using database, and we choose Mongodb as our first choice. It is safer than the way we current used, and it could be easily accessed in JS.


	\section{Tests}
	\textbf{Integration Tests} \\
	Based on our design, we make 6 integration tests to see how our program perform. The tests include the general format of the interface, dynamically display the books’ information and search results, log in status, display user / seller’s information, modify user / seller’ information, and add / delete books to the cart. Below we will test the program one by one accordingly and list the results.


	\subsection{Test 1: Interface format}

	\textbf{Test design:} \\
	Because user of the program could use tools with different size of screen to access the program, so it should accordingly display its content. The test we performed here is show the display is successes over different size of screen.

	\textbf{Test execution:} \\
	When user go to the page, they should be able to view all intended information, and the program should generate the format of the content according to the screen’s size.

	\textbf{Test result: } Failed

	\textbf{Conclusions:} \\
	We have set each item’s size fixed which is why the program fail the test. We could set item’s size by \% and allow scroll in the page.


	\subsection{Test 2: Dynamically display the books' information and search results}

	\textbf{Test design:} \\
	One of the main function of the program is user could look at all books or search for a book. The test will show the content is display correct and search result is correct.

	\textbf{Test execution:} \\
	When user go the main page, the program should display the main page and all books according to alphabet. When user search a book, the program should display the searched book if found, show there is no such book is not found.

	\textbf{Test result: }\\
			Display main page: Passed \\
          	Result / find book: passed \\
          	Result / not found: passed \\

	\textbf{Conclusions:} \\
	The function of display main page and show search result is correct, we plan to use template in JS instead of HTML, we test it again after the change.


	\subsection{Test 3: Log in status}

	\textbf{Test design:} \\
	The page’s function is tide with user’s status like add book to cart or buy books, so the program should recognize user’s name and show wither the user is login.

	\textbf{Test Execution: } \\
	When user go to main page, they could optional login. when user as guest, there is the option to login. After logging in, the correspond location will display user’s name. Also, if the user fails the log in, the program will show error message and return to main page and show status not log in.

	\textbf{Test result: } Passed \\

	\textbf{Conclusions:} \\
	We will later move user’s information to Mongodb, after that, we will re-test it to see if the part is still worked.


	\subsection{Test 4: Display user / seller's information}

	\textbf{Test design:} \\
	The program should keep user information after register and then allow user to log in the account. Also, after logging in, the user could ask for stored information.

	\textbf{Test execution:} \\
	When user log in, they could click the icon in the main page to show their personal information, then the program will go to the correspond page, which the information is linked to user’s name.
	\textbf{Test result: } Passed \\

	\textbf{Conclusions:} \\

	We will later move user’s information to Mongodb, after that, we will re-test it to see if the part is still worked.


	\subsection{Test 5: Modify user / seller's information}

	\textbf{Test design:} \\
	The user’s information is stored in the program, and user also have the access to the stored information and permission to change it. The test is for if the user could change and re-store their personal information.

	\textbf{Test execution:} \\
	When user in the personal information page, there is an option to edit their information and make the information editable, when finish the edit, click save to re-store the information. The program should return status information for the action.
	\textbf{Test result: } Passed \\

	\textbf{Conclusions:} \\
	The current program is satisfying for this function, will re-test it after any changes.


	\subsection{Test 6: Add / delete books to/from cart}

	\textbf{Test design:} \\
	The program should be able to keeps user’s information as well as their shopping list (cart), after logging in, in any page the user should be able to check their cart by click the icon. The test is to see if the program shows the cart’s content correct in current status or after any change (add / delete).

	\textbf{Test execution:} \\
	When the user login, check the cart, and then add one book, check cart, then delete the book in the top and check the cart.
	\textbf{Test result: } Passed \\

	\textbf{Conclusions:} \\
	The cart’s information is stored locally for now, and the program could access and modify it. After any changes we will test it again.



	\section{Meeting Report}

		\subsection{Schedule for Next Week}
		

		\subsection{Progress Made}
		This week we worked more on our project and managed to (mostly) implement user
		stories #4 (remove textbook from cart) and #10 (seller can make book postings).
		We divided this project somewhat differently than for prior projects. We had some
		group members focus moreso on documentation and then had the other members work on
		implementing the user stories we had planned for this week.

		\subsection{Plans/Goals for Next Week}
		Next week, we will 

		\subsection{Contributions of Each Member}
			HTML--Andrew, Jacob, Cong \\
			JavaScript--Jacob \\
			Product Release--Jacob \\
			Refactoring--Mingyu           \\
			Design Changes and Rationale-- \\
			User Story--Cody\\
			Tests--Mingyu \\
			LaTeX--Jacob \\
			Meeting Report--Jacob \\


		\subsection{Customer}
		Customer was understanding of some realistic goal changes we had to make this week. We narrowed the scope of what to accomplish this week slightly as members’ were busier than expected. Also, we perhaps underestimated the time investment required for the initial setup of the website.

	\end {document}
	

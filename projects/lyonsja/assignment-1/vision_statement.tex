\documentclass[12pt]{article}
\usepackage{url}




\title{Textbook Buyback App--Fairbook}

\author{Cong Yang, Jacob Lyons}



\begin{document}
	\maketitle

	\tableofcontents
                \section{Info}
		Team Name: Fairbook \\
		Cong Yang ONID: yangco \\
		Jacob Lyons ONID: lyonsja \\


	\section{Problem}
	The price of textbooks continues to rise unreasonably fast. However, publishers aren't the only ones taking advantage of students. 
	Existing textbook buyback options for students are sorely lacking. With the demise of Half.com, 
	no good replacement has appeared for students looking to sell their textbooks quickly--and without 
	getting ripped off.	
        Most buyback programs use predatory pricing tactics that blatantly exploit students. Although independent options 
	exist (ebay, craigslist, Amazon, etc.) they are
	generalized marketplaces with no special consideration for students just trying to sell books. For example,
	ebay places limits on how many items new sellers can list. Amazon has a more involved process before allowing accounts to start selling items. For students just trying to sell a few books,
	there are almost no good options.
	
	When the time comes to sell back textbooks, students are mostly presented with options heavily weighted
	against them. My friend has been unable to sell her textbook for a fair price, which was only published 8 months prior.
	All of the services she has looked into are offering only a tenth of the book's actual value. The bookstore pays only
	a fractional value of the textbook as do the buyback tents that appear seasonally.
		

	\section{Solution}

	The solution is to create a new web (and/or mobile) app that addresses the issues with existing platforms. Mainly the
	exploitave pricing and lack of a dedicated, local textbook buying/selling service.
	It will ideally target the college student demographic as a robust, accessible app. Utilizing existing online
        information on textbook prices (Amazon, ebay, etc), it would have \underline{fair} pricing for students. This element would be the
	key difference between this app and other services.
	It should have functionality for students to add pictures, title, and condition of the book they are trying to sell.
	The app should also have functionality for students to connect locally and meet up in person to sell textbooks.
        Also, it would allow for students to sell books to distant buyers as well.
	The main interface components would be for buyers and sellers. Interface should be intuitive, responsive,
        yet minimalistic. Another required feature would be search and ability to filter books by distance, condition, price, etc.


	\section{Risks}
	The risks of this implementation are several. Most importantly, marketing could present a significant obstacle.
        Even if the app itself is well-developed, it will require a large user base in order to be effective. The local
	buying/selling portion of the service should help it get off the ground.	
	Perhaps some sort of initial incentive could be afforded for early users. 



\end {document}

